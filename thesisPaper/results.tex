\chapter{Results}
\label{res}

\section{Methodology}
To determine the performance of the Parallel-J solutions to the problems listed in Chapter \ref{probs}, 
we developed for each problem both a sequential and parallel solution. 
The differences between these two solutions consists of nothing more than a boolean flag 
to turn parallelism on or off. 
Additionally, the results of solutions in C with OpenMP for the same problems 
are given for comparison.

All solutions were compiled and run on the Dione00 work station. %TODO get machine specs!
Before the performance of the Parallel-J solutions were recorded, 
these programs were run 5 times in order for the Java Just-In-Time compiler %TODO cite?
to make additional optimizations.
All solutions were then run 5 times then averaged, with performance measured in milliseconds.
The software used to measure the performance of the Parallel-J solutions was the Scala API Benchmark object, % TODO cite
which forces garbage collection between every program run.
Performance measurement code was written for each of the C solutions.

Finally, each of the Parallel-J solutions are listed in Appendix \ref{apsol}. % TODO list benchmark, refactor

\section{Calculating Pi using Numerical Integration}
The number of operations performed in this problem scales 
as the number of divisions $n$ between the interval $[0,1]$ grows.
For our tests, we chose $n=20000$.

Table \ref{tnip} below gives the outcomes.

\begin{table}[htbp]
\begin{center}
\begin{tabular}{|r|l|l|l|l|l||l|}
	\hline
	           & Trial 1 & Trial 2 & Trial 3 & Trial 4 & Trial 5 & Average \\ 
	\hline
	\hline
	Sequential & 21995   & 19815   & 19908   & 19713   & 19717   & 20229.6 \\ 
	\hline
	Parallel   & 23557   & 24568   & 24497   & 24256   & 24145   & 24204.6 \\ 
	\hline
\end{tabular}
\end{center}
\caption{A nice caption goes here.}
\label{tnip}
\end{table}

\section{Game of Life}
For the two-dimensional version of this problem, 
the number of operations performed in this problem scales 
as the dimensions $x$ and $y$ grow larger, 
and also as the number of desired iterations $i$ grows. 
The latter, however, is fundamentally sequential, 
so $i$ was fixed to be relatively small with value 10, 
whereas $x$ and $y$ were both set to 100, 
meaning each iteration consists of updating 10,000 cells.

Table \ref{tgol} below gives the outcomes.

\begin{center}
\begin{tabular}{|r|l|l|l|l|l||l|}
\label{tgol}
	           & Trial 1 & Trial 2 & Trial 3 & Trial 4 & Trial 5 & Average \\ \hline
	Sequential & 72512   & 68714   & 68679   & 68604   & 68611   & 69424   \\ \hline
	Parallel   & 70105   & 68092   & 67755   & 67743   & 67541   & 68247.2 \\ \hline
\end{tabular}
\end{center}

\section{Merge Sort}
The number of operations required for this problem 
grows as the number $n$ of values to sort increases. 
Because of the problem constraint listed in Section \ref{mgdes}, 
we chose $n=16384$, which is $2^{14}$.
These are all 32-bit integers; 
in the Parallel-J example, they are also unique values 
in the range of $[0,16384)$. 

Table \ref{tmrg} below gives the outcomes.

\begin{center}
\begin{tabular}{|r|l|l|l|l|l||l|}
\label{tmrg}
			   & Trial 1 & Trial 2 & Trial 3 & Trial 4 & Trial 5 & Average \\ \hline
	Sequential & 6128    & 2998    & 2969    & 2849    & 2826    & 3554    \\ \hline
	Parallel   & 4158    & 2560    & 2588    & 2540    & 2550    & 2879.2  \\ \hline
\end{tabular}
\end{center}
