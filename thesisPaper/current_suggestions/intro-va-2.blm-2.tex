\chapter{Introduction}

\section{Motivation}
Many scientific and business computing applications must work on large data sets naturally structured in regular, multidimensional collections.
In order for these applications to achieve good % (VA) maximum? By "good" you mean what? You might want a more focused term
% (BLM) I think "good" is okay in context -- readers with enough background to understand the thesis know what it means, and stronger words such as "maximum" or "optimum" don't seem to me to be meaningful here
performance, it is often the case that programmers must exploit any and all concurrency % (VA) do you need to define this?
% (BLM) Not for likely readers, though I'd say "potential concurrency", here and in the following text.  
in the application through different approaches to parallel programming.
One approach frequently used to exploit concurrency on large collections is \textit{data parallelism}, in which programs update elements or subsections of collections in parallel.
% (BLM) This sentence could be wordsmithed some.  The point I would want to make is that in programs that operate on large collections of data the potential concurrency is often of the "data parallelism" form.
Many tools have already been developed to help programmers exploit data parallelism easily and safely. %TODO cite OpenMP, dph (VA) what do you mean by "safe"?
% (BLM) Here I agree that it would be helpful to say what you mean by "safely"
However, we believe that many of these tool are % (VA) may not be
% (BLM) I say go with "are" since you do say "we believe" -- but perhaps VA has a point.
not as helpful for exploiting data parallelism on regular collections as they could be, % (VA) but is this a problem? why?
% (BLM) Interesting question, but not one I can really imagine a likely reader asking -- my reaction is "well of course it's a problem!" 
for two reasons.

First, many programming languages implement regular collections as nested collections;
e.g. a 2-dimensional, 2 by 3 array (or rank 2 array with shape 2 3) of integers would be implemented as 
an array of 2 arrays, with each of these containing 3 integers. % TODO mention alternative of shape vector associated
However, in such languages irregular collections are also usually implemented as nested collections, 
with each of the sub-collections possibly containing a different number of elements from each other.
Since both regular and irregular collections are implemented the same way,
it can be difficult for programmers to view, manipulate, and verify any important properties of the structure of regular collections.
For example, if a function requires that one or more of its arguments is a regular collection, 
programmers must either write code to ensure this requirement is met % (VA) why is this a problem?
% (BLM) because the alternative, which you state next, is obviously(?) bad -- this seems clear to me and I think to most programmers
or risk run-time indexing errors and program crashes.

Alternatively, regular collections can be implemented as vectors which have an associated shape vector. 
Making the structure of a regular collection a field which can be viewed and manipulated 
allows the programmer greater transparency and control when writing programs to operate on them.
Unfortunately, most statically-typed languages do not have type systems advanced enough 
to capture all of the structural information of regular collections statically, even though this is theoretically possible. % TODO cite dependent types (xi probably)
However, some languages are able to capture some of this structure statically, \cite{boost} \cite{sac} \cite{dph}
enabling programmers to reduce or eliminate the need for manual verification, 
such as the fact that the arguments to a function are in fact regular collections or lower bounds to the ranks of a function's argument.
Even when these properties of regular collections cannot be captured statically, making this information more transparent at run time 
should allow programmers to more easily verify that the required properties hold.

Second, in most imperative languages with regular collections, 
applying functions to elements or sub-collections of a collection means 
writing the function call within nested looping structures, typically a \textit{for loop}.
Using this method to apply functions on these collections usually means that 
the number of dimensions (also called \textit{data rank}) of the collection 
determines the number of loops required to do a specific operation.
To put it another way, if an existing function that operates on a multidimensional collections 
needs to be extended to a collection of higher dimensions, 
or if the function only operates on scalar values but needs to be extended to some multidimensional collection, 
the programmer usually must wrap the function in nested for loops.
Without abstraction in the language or library to deal with the general cases of such extensions, 
this activity is tedious % (VA) I guess I see why "tedious" is something one wants to avoid, but is it a problem in itself? In other words, many tasks are tedious but important. Do you see what I'm getting at?
% (BLM) If the tedium is somehow inherent in the important task, yeah, kokay, but part of what programmers *do* is look for ways to reduce tedium, no?
in the trivial cases and prone to unnecessary error 
because programmers must instead write boilerplate code % (VA) how exactly does this lead to error
% (BLM) I think the relationship here will be moderately obvious to someone who knows about programming. 
in order to accomplish it.
Moreover, in some cases these extensions are data parallel, 
and can be done automatically and automatically exploit inherent concurrency.
Using nested for loops obfuscates this inherent concurrency and 
requires the programmer to make changes to existing code in order to exploit it. % (VA) "and thus...". Is this a problem?
% (BLM) Yes, it's a problem.  To me it seems obvious that if you can take advantage of the possible concurrency without changing the programmer that's a plus.  But I guess you could put in something, maybe in parentheses, about how it seems better in general to express the potential concurrency directly rather than turning it into an apparently-sequential loop and hoping the compiler can reverse this transformation.  Not an original thought but a good one.

Functional programming languages, in contrast to imperative programming languages, 
allow programmers to view and write programs at a higher level of abstraction. % (VA) higher than what (CJ) her comment was there before "imperative" was added
% (BLM) Higher than imperative, of course.  I'd agree that the way you've worded doesn't make that 100% clear but I don't know how to do it better.  Maybe VA does.
When dealing with collections, this higher level of abstraction usually means that related types of operations can be expressed as \textit{higher-order} functions 
like \textit{map} and \textit{reduce} that take functions as their arguments and return functions that operate on collections.
These higher-order functions can allow programmers to more easily understand what operations are being done to multidimensional collections 
than if the same operations were done imperatively using for loops 
because they capture the essence of what the operations \textit{are}, % (VA) do you need to prove this claim?
% (BLM) I don't think so.
more than how the operations are \textit{implemented}.
As a consequence, this % (VA) what exactly?
% (BLM) The higher-order functions?  so maybe "they" instead of "this"?
also allows programmers to more easily see %(VA) avoid split infinitives; "to see more easily"
% (BLM) Ah, the split-infinitives dispute ....  Certainly there are language pedants who object to "split infinitives" (quotations because -- eh, sometime in person maybe), but there are also language pedants who don't.  If you're writing for one of the former I suppose it's a courtesy to cater to their preferences, but otherwise I say do what you think makes the meaning clearest.
where there is inherent data parallelism in the algorithm, such as all calls to \textit{map}, or calls to \textit{reduce} with associative operations, on large collections.
However, since these higher-order functions are usually designed to return functions that operate on only one dimension at a time, 
they too must be nested in order to extend existing functions to collections of higher dimension, with the same problems mentioned above. % (VA) "and thus"... you need to bring this prg. to a close
% (BLM) You do?  I'll defer to the language/rhetoric expert, but to me this seems okay in context, though *possibly* you could be more explicit about what problems you mean (rather than "mentioned above").

\textit{Function rank}, first introduced by K. Iverson in 1978\cite{opandfunc} %(VA) good transition
and implemented in the programming language J, 
is a functional abstraction that extends the notion of data rank to functions. 
In functional languages where the idea of function rank is formalized, 
extending existing functions to regular collections of higher dimension 
can be expressed as the application of a higher-ordered function, 
called in J and in this paper the \textit{rank operator}. \cite{jvocab} %(VA) punctuation, space
% (BLM) No idea what VA means here, but as I think I said in a previous comment (possibly not yet reviewed) I think you need to put the \cite before the punctuation, here and elsewhere.
Expressing these extensions as a higher-order function 
makes it easier for a programmer to make them %(VA) which you'll need to demonstrate
safely, %(VA) ?
% (BLM) If you explain earlier what you mean by "safely" I say no need to repeat it here, and ....  I think earlier you explained, or tried to, why the alternative -- loops -- is problematical, so no need to prove this claim.
quickly, and at a higher level of abstraction.
In some cases, these extensions are so trivial that they can be done automatically,\cite{jvocab} \cite{rankanduni} 
meaning the programmer need not modify the code at all. 
Furthermore, since multiple applications of the rank operator 
are equivalent to the cases where nested loops or nested higher-order function applications are inherently parallel, 
it too is inherently data parallel.
Consequently, languages with both formalized function rank and a rank-operator allow the programmer to 
exploit the inherent data parallelism of extending existing operations to collections of higher dimension 
safely, quickly, and in some cases automatically.

% (VA) do you need to bring in secondary support/document these claims
% (BLM) the ones in the previous paragraph?  I think not.  I suppose there could be some programmers who'd disagree, but I'm not sure this is something you can really prove one way or another, and, I don't know, a lot of what you're saying seems to me like "conventional wisdom" to enough programmers to not need explicit support.  (Split infinitive!)
% (BLM ROUND 2) This is the place where I misunderstood, right?  and the cquestion was about the following paragraph?  assuming so ....  continued at the end of the paragraph.
However, currently the languages that meet these criteria, 
such as J, Sharp APL, and some others from the APL language family, 
are not in common use. % TODO cite lack of parallelism
One often-cited reason for this is that these languages are difficult to read, % TODO cite
because, in order to use them effectively, a programmer must memorize 
dozens of 1 or 2 character functions each with different, sometimes unrelated use cases 
depending on whether the function takes one or two arguments.\cite{jvocab} %TODO cite APL cheat sheet
However, these language design choices are not required in order for a language to support function rank.
A proof-of-concept is needed % (VA) may be needed
% (BLM) Eh, maybe.  Maybe better to just reword, something like "In support of this claim we present extensions to J and a proof-of-concept implementation ...." ?
to demonstrate that the notion of function rank % TODO what to call shape vectored arrays?
is still very helpful for exploiting data parallelism on regular collections 
when available in a more modern, more popularly supported language.
% (BLM ROUND 2) I say you don't really need to cite sources for any of this -- to me it seems like "conventional wisdom among programmers".  The phrasing "often-cited" does perhaps suggest that there should *be* citations, but -- nah, I think it's okay.  Or you could track down a source for the joke about how one can write an APL program to [something, I forget] in a single line of code, no character of which appears on a standard keyboard.  Sort of a :-).

The rest of the paper is organized as follows:
\begin{itemize} 
	\item The remaining sections of this chapter give the design plan (\ref{desp}) and implementation (\ref{imp}) of our research. 
	Unfortunately, the latter did not quite fulfill the expectations we had in the former, so we include both a description of what we wanted to accomplish and what we were actually able to accomplish given the time constraints. 
	\item Chapter \ref{back} gives the necessary background information for understanding this work, 
	including a brief description of function rank and how it is equivalent to nested loops or nested calls to \textit{map},
	a discussion of relevant parallel design patterns for the example problem set, 
	and a literature review of other attempts to solve the same or similar problems
	\item Chapter \ref{paraop} proposes two new operators for future parallel implementations of the J programming language 
		and illustrates how these operators might be used
    \item Chapter \ref{probs} gives the listing of our selection of example problems, giving briefly for each a description, a discussion of the relevant parallel design patterns for exploiting concurrency, and a description of how possibly to extend the problem to higher dimensions.
    \item Chapter \ref{res} compares example solutions to the problems listed in Chapter \ref{probs} in Scala, J, Parallel-J and in some cases C with OpenMP, and discusses the relative level of abstraction, scalability and performance of each solution. %(CJ) TODO remove falsehoods - probably won't compare with all
    \item Chapter \ref{conc} presents our conclusions of this research and discusses future work. 
\end{itemize}

\section{Design Plan}
\label{desp}
The goal of this research was
to implement a parallel subset of the programming language J, called \textit{Parallel-J}.
This language subset would require:\cite{ioj} 
\begin{itemize}
	\item a J language lexer and parser
	\item limited memory management
	\item a look-up table for predefined and user functions
	\item a read-evaluate-print-loop (REPL), and
	\item a subset of the J primitives, including: 
	\begin{itemize}
		\item global variable assignment
		\item most array operations (creating, indexing, restructuring)
		\item all of the basic arithmetic and logical operations
		\item a limited selection of higher-ordered functions (composition, conditional, reduction)
		\item J's function composition rules for sequences of functions (\textit{trains}), and
		\item the rank operator and function rank
	\end{itemize}
\end{itemize}

We excluded from the design of the parallel language subset the following language features: 
\begin{itemize}
	\item namespaces (\textit{locales})
	\item all existing environment-altering functions (the \textit{foreign} operator, spelled \ttfamily!:\normalfont)
	\item explicit scripts with imperative-style execution and control structure, and
	\item the J primitives which calculate complex algorithms (prime factorization, derivatives of polynomial equations, etc)
\end{itemize}

These choices were motivated by the desire to present solutions to our sample problems in as simple and ``idiomatic'' J as possible.
This would allow us to directly address the involved computations, 
as well as the advantages to exploiting data parallelism when approaching these problems with function rank, 
without getting distracted with a discussion of J's more advanced or nuanced language features.

Also to be included was a proposal and implementation of a new \textit{foreign} operator, called the \textit{parallel operator} and spelled\ttfamily (111 !:)\normalfont , 
that would allow the programmer to set environmental parameters for parallelizing code.
For example, this operator could be used to specify the number of threads to use in executing a piece of code, 
or to set thread scheduling schemes.
Although not entirely related to function rank, such controls are to be expected from any serious parallel computing environment.

To implement this parallel subset of J, we chose the Scala programming language.\cite{scala} 
Scala has many features that make it desirable for this task, including: 
\begin{enumerate} 
	\item support for multiple programming paradigms, such as
	\begin{itemize}
		\item imperative, for the tasks of memory management and object creation,
		\item object oriented, for more structured encapsulation of the rank associated with each function 
			and data associated with arrays, and
		\item functional, to more easily capture J's functional paradigm in the implementation and thus facilitate development
	\end{itemize}
	\item a feature-rich library for collections, including 
	\begin{itemize}
		\item support for many higher-order operations such as \textit{map} and \textit{reduce}, \cite{scala28col}
			common to many solutions to data parallel problems, and
		\item a library for parallel collections that exploits the concurrency inherently data parallel operations,\cite{pc}
			which, being similar to our own use cases, we anticipated would make parallelizing our implementation relatively easy.
	\end{itemize}
\end{enumerate}

We would then compare solutions to a suite of data parallel problems written in C with OpenMP, Scala and Parallel-J,
and discuss the relative level of abstraction, scalability to similar problems of higher dimension, and performance of each. 
It was not expected that Parallel-J would compare favorably to the raw performance of C or the current implementation of J; 
the hoped for result was to show that the performance Parallel-J scales relative to the number of threads, 
showing that this implementation does effectively exploit potential concurrency.
% TODO also parallelisation techniques.

\section{Implementation}
\label{imp}
Unfortunately, during the development of this research it became clear that the given time constraints would not allow for such an ambitious project.
This has lead to a change of focus in the research. 
Rather than finishing a parallel subset of J, the J system libraries developed 
are instead used as a prototype for a Scala regular collections library which supports function rank.
Taking the same set of parallel problems, we show both how this library allows for uniformly extending these problems into higher dimensions and 
how with future work it would be able to automatically parallelize solutions in the given and in higher dimensions.
We also compare our solutions with solutions in J and in C with OpenMP, discussing relative level of abstraction, scalability, and performance of each.

The proposal for the \textit{parallel operator} is still given but it unfortunately could not be implemented in the time available.

Finally, some work completed with the idea of implementing a subset of J did not contribute to developing a regular collections library in Scala that supports function rank.
Most of this work consisted of implementing a lexer for the J programming language, which is listed in Appendix \ref{jlex}
