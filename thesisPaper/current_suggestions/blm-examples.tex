\chapter{Tests}

\section{Code set with verbatim}
Below is an indented inline code example.

\begin{quote}
\begin{singlespacing}
\begin{small}
\begin{verbatim}
   1 + 1
2
   show =: ] NB. Identity, used to display results
   integers =: i. NB. creates array with shape of argument
   NB.                populated with an incrementing value
   NB.                starting at 0
   show mat2_3 =: integers 2 3
0 1 2
3 4 5
   from =: { NB. Indexing into an array
   NB.           expressed as a function
   1 from mat2_3
3 4 5
\end{verbatim}
\end{small}
\end{singlespacing}
\end{quote}

\noindent
In some cases, when the arguments to \texttt{f} do not match its function rank, \texttt{f} is automatically extended to the appropriate dimensions.
For example, if \texttt{x} is a scalar, addition can always be extended so that \texttt{x} is added to every element of a collection \texttt{c} 
no matter \texttt{c}'s rank or shape. 

Now an unindented code example.

\begin{singlespacing}
\begin{small}
\begin{verbatim}
   1 + mat2_3
1 2 3
4 5 6
\end{verbatim}
\end{small}
\end{singlespacing}

\noindent
Code in Figures~\ref{fig:code1} through \ref{fig:code4}
illustrates something.
% 
% commented-out whitespace -- don't want to start a new paragraph
% but might want whitespace in LaTeX source
%
% usually want \begin{figure}[htbp] but this example is long so put
% each piece on a page by itself
\begin{figure}[p]
\begin{quote}
\begin{singlespacing}
\begin{small}
\begin{verbatim}
   100 200 + mat2_3
100 101 102
203 204 205

   NB. agreement: visualizes
   NB. how the cells of each 
   NB. collection are paired with each other
   NB. before performing the desired operation
   agreement =: ; "

   NB. Show agreement of two collections above
   NB. under adition
   NB. The shape 2 is the frame;
   NB. the scalars are expanded to
   NB. vectors of 3 to match 
   NB. the shape of mat2_3
   100 200 (+ agreement) mat2_3
+---+-+
|100|0|
+---+-+
|100|1|
+---+-+
|100|2|
+---+-+

+---+-+
|200|3|
+---+-+
|200|4|
+---+-+
|200|5|
+---+-+
\end{verbatim}
\end{small}
\end{singlespacing}
\end{quote}
\caption{A nice caption here.  
Example continues in Figure~\ref{fig:code2}.}
\label{fig:code1}
\end{figure}
%
\begin{figure}[p]
\begin{quote}
\begin{singlespacing}
\begin{small}
\begin{verbatim}
   mat2_3 + mat2_3
0 2  4
6 8 10
   NB. The frame is 2 3;
   NB. the scalar cells of both collections
   NB. are paired with each other
   mat2_3 (+ agreement) mat2_3
+-+-+
|0|0|
+-+-+
|1|1|
+-+-+
|2|2|
+-+-+

+-+-+
|3|3|
+-+-+
|4|4|
+-+-+
|5|5|
+-+-+
\end{verbatim}
\end{small}
\end{singlespacing}
\end{quote}
\caption{A nice caption here.  
Example continued from Figure~\ref{fig:code1},
continuing in Figure~\ref{fig:code3}.}
\label{fig:code2}
\end{figure}
%
\begin{figure}[p]
\begin{quote}
\begin{singlespacing}
\begin{small}
\begin{verbatim}
show arr2_3_2 =: integers 2 3 2
 0  1
 2  3
 4  5

 6  7
 8  9
10 11
   arr2_3_2 + mat2_3
 0  1
 3  4
 6  7

 9 10
12 13
15 16
   NB. The frame is 2 3;
   NB. The scalar cells of mat2_3
   NB. are expanded to vectors of 2
   NB. to match the shape of
   NB. arr2_3_2
\end{verbatim}
\end{small}
\end{singlespacing}
\end{quote}
\caption{A nice caption here.  
Example continued from Figure~\ref{fig:code2},
continuing in Figure~\ref{fig:code4}.}
\label{fig:code3}
\end{figure}
%
\begin{figure}[p]
\begin{quote}
\begin{singlespacing}
\begin{small}
\begin{verbatim}
   arr2_3_2 + agreement mat2_3
+--+-+
|0 |0|
+--+-+
|1 |0|
+--+-+

+--+-+
|2 |1|
+--+-+
|3 |1|
+--+-+

+--+-+
|4 |2|
+--+-+
|5 |2|
+--+-+


+--+-+
|6 |3|
+--+-+
|7 |3|
+--+-+

+--+-+
|8 |4|
+--+-+
|9 |4|
+--+-+

+--+-+
|10|5|
+--+-+
|11|5|
+--+-+
\end{verbatim}
\end{small}
\end{singlespacing}
\end{quote}
\caption{A nice caption here.  
Example continued from Figure~\ref{fig:code3}.}
\label{fig:code4}
\end{figure}
%
More text after the figures.  (Not a new paragraph.)

\section{Math and code set as math}

Equation~\ref{eqn:famous} is well-known.
words words words words words words words words 
words words words words words words words words 
words words words words words words words words 
words words words words words words words words 
%
\begin{equation}
E = mc^2 \label{eqn:famous}
\end{equation}
%
Here are examples of code set as unnumbered equations.
words words words words words words words words 
words words words words words words words words 
words words words words words words words words 
words words words words words words words words 
%
\[ \texttt{(sh.:Int.:Int.:Int)} \]
%
words words words words words words words words 
words words words words words words words words 
words words words words words words words words 
words words words words words words words words 
%
\[ \texttt{f :: (Shape sh, Elt e) => Array (sh :. Int :. Int :. Int) e -> Array sh e} \]
%
words words words words words words words words 
words words words words words words words words 
words words words words words words words words 
words words words words words words words words 
