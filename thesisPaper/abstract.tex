\begin{center}

\bigskip

\begin{Large}
\textbf{\theTitle}
\end{Large}

\bigskip

\begin{large}
\theAuthor
\end{large}

\bigskip
\bigskip

\textbf{Abstract}

\end{center}

\noindent
Many scientific and business applications operate on 
data conceptually contained in regular, multidimensional collections.
Frequently, operations on these collections are inherently data parallel.
However, most languages and parallel environments 
provide tools for exploiting data parallelism only for collections of one dimension.
Programmers who wish to use such tools on existing programs to operate on regular, multidimensional collections
often must write non-generalized code that can be difficult to maintain and 
that cannot automatically scale to collections of higher dimensions.

Function rank, first developed by K. Iverson and implemented in the programming language J, 
formalizes the dimension (\textit{rank}) of the arguments to a function in such a way 
that functions can be easily extended to collections of higher dimensions using 
a higher-order function (the \textit{rank operator}).
This extension is relatively simple, inherently parallel, and provably safe 
from certain kinds of indexing errors which stem from not knowing the dimension, shape, or possible irregularity of the collection.
However, this last property can be difficult to capture at compile-time.

Presented is a prototype for a parallel regular collections library that supports function rank 
to demonstrate that function rank allows a programmer to easily exploit potential concurrency 
in problems involving inherently data parallel operations on regular collections.
The programming language Scala was selected to implement this library because it is modern, multi-paradigm, 
and in particular it provides a parallel collections framework that 
allows programmers to easily write data parallel programs for 1 dimensional collectons.
Also presented are three proposals that could be included in 
possible future parallel implementations of the J programming language:
two new operators, parallel rank and parallel insert;
and an extension to the J environment library to include functions which affect a parallel environment.
Each proposal includes a sketch of its use.

Finally, we use a suite of example problems, 
solved in J, C using OpenMP, and in Scala using the presented library, 
to compare conceptual ease of use and scalability of the applications to related problems in higher dimensions.
Each problem is discussed in the context of wider parallel design patterns, 
to show that function rank can be useful to a broad class of parallel problems.
Performance results are included, but do not show much significance in improved performance 
due to a lack of optimization; this work is intended as a proof-of-concept for future research.
