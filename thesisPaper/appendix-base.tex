\chapter{J \texttt{\#:} (Base) Primitive}
\label{base}
The J primitive \texttt{\#:} is a function that allows 
programmers to easily convert numeric values into different number bases. 
For its one argument (\textit{monadic}) usage, 
the primitive defaults to converting values into base 2. 
The result of applying this function as a monad 
is a vector of 0s and 1s, with the indicies giving the appropriate place value.

\begin{singlespacing}
\begin{small}
\begin{verbatim}
   base =: #:
   base 2
1 0
   base 3
1 1
   base 65
1 0 0 0 0 0 1
\end{verbatim}
\end{small}
\end{singlespacing}

For its two argument (\textit{dyadic}) use case, \texttt{\#:} takes as its left argument 
a vector representing the place values of the desired base, shown below. 

\begin{singlespacing}
\begin{small}
\begin{verbatim}
   2 2 base 3
1 1
   2 2 2 2 2 2 2 base 65
1 0 0 0 0 0 1
   10 10 base 65
6 5
\end{verbatim}
\end{small}
\end{singlespacing}

One advantage of taking a vector for a numeric base representation, instead of a more traditional scalar (base 2, 8, 10, 16, etc.), 
is that that \texttt{\#:} can also represent numbers in irregular numeric bases.
The example below shows how \texttt{\#:} can be used to represent 100,000,000 milliseconds in 
the most familiar of the irregular numeric bases, time as measured in days, hours, seconds, and milliseconds.

\begin{singlespacing}
\begin{small}
\begin{verbatim}
   365 24 60 60 1000 base 100000000
1 3 46 40 0\end{verbatim}
\end{small}
\end{singlespacing}

In the discussion of extending the Game of Life to higher dimensions in Section \ref{piext}, 
the dyadic case of \texttt{\#:} is used in the function \texttt{xs}, 
which takes a vector \texttt{vn} of length \texttt{n} specifying how to divide the interval $[0,1]^n$. 
\texttt{\#:}'s purpose in \texttt{xs} is to convert scalar values to 
a vector \texttt{vcn} representing coordinate values, where each scalar \texttt{c} at index \texttt{i} in 
\texttt{vcn} lies in the interval $[0, \texttt{i from vn}]$.
Finally, each vector element in \texttt{vcn} is divided by \texttt{vn} so that the resulting coordinate values 
all lie in the interval $[0,1]^n$.
This process is illustrated step by step in the following example.

\begin{singlespacing}
\begin{small}
\begin{verbatim}
   intergers =: i.
   insert =: /
   (* insert) 4 4
16
   integers (* insert) 4 4
0 1 2 3 4 5 6 7 8 9 10 11 12 13 14 15
   4 4 base integers (* insert) 4 4
0 0
0 1
0 2
0 3
1 0
1 1
1 2
1 3
2 0
2 1
2 2
2 3
3 0
3 1
3 2
3 3
   (4 4 base integers (* insert) 4 4) %(" 1) 4 4
   0    0
   0 0.25
   0  0.5
   0 0.75
0.25    0
0.25 0.25
0.25  0.5
0.25 0.75
 0.5    0
 0.5 0.25
 0.5  0.5
 0.5 0.75
0.75    0
0.75 0.25
0.75  0.5
0.75 0.75
\end{verbatim}
\end{small}
\end{singlespacing}
