\chapter{Introduction}

\section{Motivation}
Many scientific and business computing applications must work on large data sets naturally structured in regular, multidimensional collections.
In order for these applications to achieve good performance, it is often the case that programers must exploit any and all concurrency in the application through different approaches to parallel programming.
One approach frequently used to exploit concurrency on large collections is \textit{data parallelism}, in which programs update elements or subsections of collections in parallel.
Many tools have already been developed to help programmers exploit data parallelism easily and safely. %TODO citation
However, we believe that many of these tool are not as helpful for exploiting data parallelism on regular collections as they could be, for two reasons.

First, many programming languages implement regular collections as nested collections;
e.g. a 2-dimensional, 2 by 3 array (or rank 2 array with extent 2 3) of integers would be implemented as an array of 2 arrays, with each of these containing 3 integers. % TODO mention alternative of shape vector associated
However, in such languages irregular collections are also usually implemented as nested collections, with each of the sub-collections possibly containing a different number of elements from each other.
Since both regular and irregular collections are implemented the same way,
it can be difficult for programmers to view, manipulate, and verify any important properties of the structure of regular collections.
For example, if a function requires that one or more of its arguments is a regular collection, programmers must either write code to ensure this requirement is met or risk run-time indexing errors and program crashes.

Alternatively, regular collections can be implemented as vectors which have an associated shape vector. 
Making the structure of a regular collection a field which can be viewed and manipulated 
allows the programer greater transparency and control when writing programs to operate on them.
Unfortunately, most statically-typed languages do not have type systems advanced enough 
to capture all of the structural information of regular collections statically, even though this is theoretically possible. % TODO cite dependent types (xi probably)
However, some languages are able to capture some of this structure statically, \cite{boost} \cite{sac} \cite{dph}
enabling programmers to reduce or eliminate the need for manual verification, 
such as the fact that the arguments to a function are in fact regular collections.
Writing effective parallel code is an already conceptually difficult task without adding manual checks for program correctness.

Second, in most imperative languages with regular collections, applying functions to elements or sub-collections of a collection means writing the function call within nested looping structures, typically a \textit{for loop}
Using this method to apply functions on these collections usually means that the number of dimensions (also called \textit{data rank}) of the collection determines the number of loops required to do a specific operation.
To put it another way, if an existing function that operates on a multidimensional collections needs to be extended to a collection of higher dimensions, 
or if the function only operates on scalar values but needs to be extended to some multidimensional collection, 
the programmer usually must wrap the function in nested for loops.
This activity is tedious in the trivial cases and prone to uneccesary error because programmers must write boilerplate code in order to accomplish it.
Moreover, even though in some cases these extensions are inherently data parallel, and can theoretically be done automatically and without changing existing code (such as applying a function on scalars), 
without abstractions in the language to deal with the general cases of such extensions, the programer must make changes to existing code.

Functional programming languages, in contrast to imperative programming languages, allow programmers to view and write programs at a higher level of abstraction.
When dealing with collections, this higer level of abstraction usually means that related types of operations can be expressed as \textit{higher-order} functions 
like \textit{map} and \textit{reduce} that take functions as their arguments and return functions that operate on collections.
These higher-order functions can allow programmers to more easily understand what operations are being done to multidimensional collections 
than if the same operations were done imperatively using for loops 
because they capture the essence of what the operations \textit{are}, more than how the operations are \textit{implemented}.
As a consequence, this also allows programmers to more easily see where there is inherent data parallelism in the algorithm, such as all calls to \textit{map}, or calls to \textit{reduce} with associative operations, on large collections.
However, since these higher-order functions are usually designed to return functions that operate on only one dimension at a time, 
they too must be nested in order to extend existing functions to collections of higher dimension, with the same problems mentioned above.

\textit{Function rank}, first introduced by K. Iverson in 1978\cite{opandfunc} and implemented in the programming language J, is a functional abstraction that extends the notion of data rank to functions. %(CJ) defined earlier in the paper
In functional languages where the idea of function rank is formalized, extending existing functions to regular collections of higher dimension can be expressed as the application of a higher-ordered function, 
called in J and in this paper the \textit{rank operator}. \cite{jvocab}
Expressing these extensions as a higher-order function makes it easier for a programmer to make them safely, quickly, and at a higher level of abstraction.
In some cases, these extensions are so trivial that they can be done automatically, meaning the programmer need not modify the code at all. %TODO cite J shape agreement
Furthermore, since multiple applications of the rank operator are equivalent to nested loops or nested higher-order function applications, it too is inherently data parallel.
Consequently, languages with both formalized function rank and a rank-operator allow the programmer to 
exploit the inherent data parallelism of extending existing operations to collections of higher dimension safely, quickly, and in some cases automatically.

Unfortunately, currently the languages that meet these criteria, such as J, Sharp APL, and some others from the APL language family, are not in common use. % TODO cite lack of parallelism
One often-cited reason for this is that these languages are difficult to read, % TODO cite
because, in order to use them effectively, a programmer must memorize dozens of 1 or 2 character functions each with different, sometimes unrelated use cases depending on whether the function takes one or two arguments.\cite{jvocab} %TODO cite APL cheat sheet
However, these language design choices are not required in order for a language to support function rank.
It seems to us that 
what is needed is a proof-of-concept that the notion of function rank is still very helpful in exploiting data parallelism on regular multidimensional arrays in a more modern, more popularly supported language. 

The rest of the paper is organized as follows:
\begin{itemize} 
	\item The remaining sections of this chapter give the design plan (\ref{desp}) and implementation (\ref{imp}) of our research. 
	Unfortunately, the latter did not quite fulfill the expectations we had in the former, so we include both a description of what we wanted to accomplish and what we were actually able to accomplish given the time constraints. 
	\item Chapter \ref{back} gives the neccessary background information for understanding this work, 
	including a brief description of function rank and how it is equivalent to nested loops or nested calls to \textit{map} and \textit{reduce},
	a discussion of relevant parallel design patterns for the example problem set, 
	and a literature review of other attempts to solve the same or similiar problems
    \item Chapter \ref{probs} gives the listing of our selection of example problems, giving briefly for each a description, a discussion of the relevant parallel design patterns for exploiting concurrency, and a descritpion of how possibly to extend the problem to higher dimensions.
    \item Chapter \ref{res} compares example solutions to the problems listed in Chapter \ref{probs} in Scala, J, Parallel-J and in some cases C with OpenMP, and discusses the relative level of abstraction, scalability and performance of each solution. %(CJ) TODO remove falsehoods - probably won't compare with all
    \item Chapter \ref{conc} presents our conclusions of this research and discusses future work. 
\end{itemize}

\section{Design Plan}
\label{desp}
Our goal was 
to implement a parallel subset of the programming language J, called \textit{Parallel-J}.
This 
would require a J language lexer and parser, limited memory management, a look-up table for predefined and user defined functions, a read-evaluate-print-loop (\textit{REPL}), and a subset of the J primitives.
To be included in this subset of primitives would be: global variable assignment; 
most array operations (creating, indexing, restructuring) and all of the basic arithmetic and logical operations; 
a small selection of higher-ordered functions, such as function composition (including J's tacit function composition rules for sequences of functions or \textit{trains}), conditionals, reduction;
and, 
of course, 
the rank operator and function rank. %TODO cite J Vocab (?) TODO itemize? TODO footnote for tacit

We excluded from the design of the parallel language subset the following language features: 
namespaces (called \textit{locales});
all existing environment-altering functions (the \textit{foreign} function);
explicit scripts with imperative-style execution and control structures;
and the J primitives which calculated complex algorithms, such as prime factorization and derivatives polynomial equations.
% TODO itemize?
These choices were motivated by the desire to present solutions to our sample problems in as simple and ``idiomatic'' J as possible.
This would allow us to directly address the involved computations, as well as the advantages to exploiting data parallelism when approaching these problems with function rank, 
without getting distracted with a discussion of J's more advanced or nuanced language features.

Also to be included was a proposal and implementation of a new \textit{foreign} operator, called the \textit{parallel operator} and spelled\ttfamily (111 !:)\normalfont , 
that would allow the programmer to set environmental parameters for parallelizing code.
For example, this operator could be used to specify the number of threads to use in executing a piece of code, 
or to set thread scheduling schemes.
Although not entirely related to function rank, such controls are to be expected from any serious parallel computing environment.

To implement this parallel subset of J, we chose the Scala programming language.\cite{scala} 
Scala has many features that make it desireable for this task, including: 
\begin{itemize} 
	\item support for multiple programming paradigms, such as
	\begin{itemize}
		\item imperative, for the tasks of memory management and object creation,
		\item object oriented, for more structured encapsulation of the rank associated with each function 
			and data associated with arrays, and
		\item functional, to more easily capture J's functional paradigm in the implementation and thus facilitate development
	\end{itemize}
	\item a feature-rich library for collections, including 
	\begin{itemize}
		\item support for many higher-order operations such as \textit{map} and \textit{reduce}, \cite{scala28col}
			common to many solutions to data parallel problems, and
		\item a library for parallel collections that exploits the concurrency inherrently data parallel operations,\cite{pc}
			which, being similiar to our own use cases, we anticipated would make parallelizing our implementation relatively easy.
	\end{itemize}
\end{itemize}

We would then compare solutions to a suite of data parallel problems written in C with OpenMP, Scala and Parallel-J,
and discuss the relative level of abstraction, scalability to similiar problems of higher dimension, and performance of each.
% TODO also parallelisation techniques.

\section{Implementation}
\label{imp}
Unfortunately, during the the development of this research it became clear that the given time constraints would not allow for such an ambitious project.
This has lead to a change of focus in the research. 
Rather than finishing a parallel subset of J, the J system libraries developed 
are instead used as a prototype for a Scala regular collections library which supports function rank.
Taking the same set of parallel problems, we show both how this library allows for uniformly extending these problems into higher dimensions and 
how with future work it would be able to automatically parallelize solutions in the given and in higher dimensions.
We also compare our solutions with solutions in J and in C with OpenMP, discussing relative level of abstraction, scalability, and performance of each.

The proposal for the \textit{parallel operator} is still given but it unfortunately could not be implemented in the time available.

Finally, some work completed with the idea of implementing a subset of J did not contribute to developing a regular collections library in Scala that supports function rank.
Most of this work consisted of implementing a lexer for the J programming language, which is listed in Appendix \ref{jlex}
