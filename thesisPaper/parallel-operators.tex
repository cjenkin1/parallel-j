\chapter{Proposed Operators for a Parallel Implementation of J} % TODO change everything mentioning the 111 !:
\label{paraop}

\section{Rationale}
Ideally, it would be the case that a program would automatically perform better using a parallel library 
without requiring the programmer to write a single line of parallelizing code. 
However, this is often not the case. % TODO cite?
Frequently, programmers must use domain knowledge of the problem or platform to achieve good results. 

In order to grant this flexibility in future parallel implementations of J, 
we propose introducing two new operators, called the \textit{parallel rank operator} 
and the \textit{parallel insert operator}.
The \textit{parallel rank operator}, described in Section \ref{prank}, 
would allow the programmer to specify the ranks on which to parallelize code.
The \textit{parallel insert operator}, described in Section \ref{pins}, 
would allow the programmer to parallelize reduction operations with associative functions.
Additionally, we propose a new system library
which would allow the programmer to give annotations or force changes in the underlying parallel environment,
described in section \ref{pfor}

The spellings $``::$ and $/::$ for the two parallel operators were chosen for mnemonic and compliance reasons.
Mnemonically, each uses the same base character as their sequential analogs, ($``$ and $/$), 
as well as two ``parallel'' colons, making it easier to remember their functions.
They also require no changes be made to the existing J lexer\cite{ioj}, 
as demonstrated below:

\begin{singlespacing}
\begin{small}
\begin{verbatim}
   jlexer =: ;:
   jlexer '+/::("1) mat2_3 +("::1)("2) arr2_2_3'
+-+---+-+-+-+-+------+-+-+---+-+-+-+-+-+-+--------+
|+|/::|(|"|1|)|mat2_3|+|(|"::|1|)|(|"|2|)|arr2_2_3|
+-+---+-+-+-+-+------+-+-+---+-+-+-+-+-+-+--------+
\end{verbatim}
\end{small}
\end{singlespacing}

For the parallel environment library, 
it seemed best to augment the existing ``foreign'' operator (spelled $!:$).
This is because the foreign operator was designed to 
allow the programmer to change environmental parameters 
such as print precision, file I/O, etc. \cite{jvocab}
We chose $111$ as the numeric encoding for the parallel environment library, again for mnemonic reasons; 
$111$ looks like three parallel lines ($11$ was already taken).

\section{Parallel Rank Operator: \ttfamily"::\normalfont}
\label{prank}

\subsection{Usage}
The proposed parallel rank operator is a conjunction 
whose first argument is a function 
and whose second argument is the ranks to both apply the function and to parallelize it. 
It would be functionally equivalent to the rank operator, i.e. 
\[f\;\;(``::r)\;y\;\;\Leftrightarrow\;\;f\;(``r)\;y\] and \[x\;\;f(``::r)\;y\;\;\Leftrightarrow\;\;x\;f(``r)\;y\] for all $f, r, x,$ and $y$.

Its purpose would be to override other parallel system defaults 
to guarantee that the resulting function would parallelize operations on sub-arrays of the given rank 
using the available threads.

To illustrate, consider the following examples:

\begin{singlespacing}
\begin{small}
\begin{verbatim}
   (increment =: >:) mat2_3
1 2 3
4 5 6
   increment"0 mat2_3
1 2 3
4 5 6
   (increment"0 link increment"1 (link =: ;) increment"2) mat2_3
+-----+-----+-----+
|1 2 3|1 2 3|1 2 3|
|4 5 6|4 5 6|4 5 6|
+-----+-----+-----+
\end{verbatim}
\end{small}
\end{singlespacing}

Incrementing numeric values always applies to scalars, so it always gives the same result, 
regardless of which rank it is applied. 
Using the parallel rank operate, this behavior would still remain; 
however, each of the cases of $increment(``::0)$, $increment(``::1)$ and $increment(``::2)$ 
would result in the environment attempting to parallelize for the scalar, row, and whole matrix, respectively.
E.g., if no limit is set on the number of threads available, 
then the system might use 6 threads for parallelizing the scalars, 
2 threads for parallelizing the vectors, 
and only 1 thread for parallelizing the entire matrix 
(which, since the example computation is so trivial, would probably lead to the best performance).

Here's a more complex example involving functions of two arguments 
as well as repeated applications of the rank operator.

\begin{singlespacing}
\begin{small}
\begin{verbatim}
   ] arr2_2_3 =: integers 2 2 3
0  1  2
3  4  5

6  7  8
9 10 11
   mat2_3 +"2 arr2_2_3
0  2  4
6  8 10

6  8 10
12 14 16
   mat2_3 +("1)("2) arr2_2_3
0  2  4
6  8 10

 6  8 10
12 14 16
   mat2_3 link"2 arr2_2_3
+-----+-------+
|0 1 2|0 1 2  |
|3 4 5|3 4 5  |
+-----+-------+
|0 1 2|6  7  8|
|3 4 5|9 10 11|
+-----+-------+
   mat2_3 link("1)("2) arr2_2_3
+-----+-------+
|0 1 2|0 1 2  |
+-----+-------+
|3 4 5|3 4 5  |
+-----+-------+

+-----+-------+
|0 1 2|6 7 8  |
+-----+-------+
|3 4 5|9 10 11|
+-----+-------+
\end{verbatim}
\end{small}
\end{singlespacing}


Using the parallel rank operator, 
$mat2\_3\;+(``::2)\;arr2\_2\_3$ would parallelize addition on each of the two matrices, whereas
$mat2\_3\;+(``::1)(``2)\;arr2\_2\_3$ would parallelize addition on each of the vector elements.

Some open questions remain, most notably 
the behavior of the parallel environment when multiple applications of the parallel rank operator are used. 
This is left as a question for future research to investigate.

\section{Parallel Insert Operator: \ttfamily /:: \normalfont}
\label{pins}
\subsection{Rationale}
Frequently, programmers need to perform some sort of reduction operation on an entire collection, 
for example finding the sum, product, maximum, minimum, etc. of a collection of numbers.
It's well known that when the reducing operation $f$ is associative, i.e. that $f(x,y) = f(y,x)$ for all $f, x, y$, 
then the reduction can be carried out in a parallel fashion with little effort on the programmer's part.
It is expedient, therefore, that future parallel implementations of J allow the programmer 
to explicitly state that the reducing operation should be done in parallel.

\subsection{Usage}
As a unary, higher ordered function (\textit{adverb}), 
the parallel insert operator would be equivalent to its sequential equivalent 
for all associative operations. 
Thus, the following J session would be the same, 
regardless of whether \ttfamily insert\normalfont was assigned to be $/::$ or $/$ .

\begin{singlespacing}
\begin{small}
\begin{verbatim}
   ] vec10 =: integers 10
0 1 2 3 4 5 6 7 8 9
   + insert vec10
45
   (greaterOf =: >.) vec10
0 1 2 3 4 5 6 7 8 9
   (greaterOf =: >.) insert vec10
9
   + insert mat2_3
3 5 7
   + insert (flatten =: ,) mat2_3
15
   + insert ("1) mat2_3
3 12
\end{verbatim}
\end{small}
\end{singlespacing}

It remains an open question how the parallel insert operator should behave on non-associative functions.
One possibility is to define its domain to be only the associative primitives. 
This would prevent logical errors such as radically different results for the same calculations on the same data, 
depending on the thread scheduling scheme. 
However, this approach would exclude the possibility of 
user-defined functions which are associative benefiting from parallelization.
Alternatively, the programmer may choose to toggle this restriction using the parallel environment library, 
discussed below.

\section{Parallel Environment Library: 111 !:} 
\label{pfor}

\subsection{Conventions of the Foreign Operator}
The foreign operator is a two-argument higher order function, or in J a \textit{conjunction}, 
whose arguments are always numeric.
Conceptually, these arguments serve as an index into system libraries and functions.
The left argument indexes the desired library.
E.g., $2$ indexes the library for functions which affect the host machine, 
$9$ indexes the library for viewing and setting global J parameters (such as print precision), and etc.
The second argument indexes a specific function.
E.g., $9!:6$ is the function which displays the print characters for J's box type.

\subsection{Usage}
This section requires more research before a full suite of library functions is proposed.
This would likely be done by reviewing the functionality of other shared-memory parallel libraries, such as OpenMP. % TODO cite
However, two types of functionality would almost certainly be required.
They are:

\begin{enumerate}
    \item Getting and setting the total number of threads available in the parallel environment,
        with an extension for J's value for infinity to mean no user-specified limit on the number of threads, and
    \item Getting and setting the thread scheduling schemes (such as static, round-robin, etc.).
        These would be encoded as numeric values, in keeping with the conventions of the foreign libraries.
\end{enumerate}

Additionally, as discussed above, it may be desirable to change 
the default behavior of the parallel insert operator, 
in order to parallelize on user-defined associative operators.

The following is not an actual session with any existing J REPL, 
but an example of how one would use this proposed library.

\begin{singlespacing}
\begin{small}
\begin{verbatim}
   (111 !: 0) '' NB. Get number of threads.
1
   (111 !: 1) 4 NB. Set number of available threads to 4.

   (111 !: 2) '' NB. Get numeric encoding of thread scheduling scheme.
0
   (111 !: 3) 1 NB. Set thread scheduling to, for example, round robin.

   (111 !: 4) '' NB. Get parallel insert restriction flag
0
   (111 !: 5) 1 NB. Enable restrictions on insert

   +/::("1) mat2_3 +("::1)("2) arr2_2_3
 6 24
24 42
\end{verbatim}
\end{small}
\end{singlespacing}

The above example would parallelize addition on the row elements of $mat2\_3$ and $arr2\_2\_3$,
then parallelize the sum operation on each of the vectors of the resulting array, 
using 4 threads with a round-robin scheduling scheme.
